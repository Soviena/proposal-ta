\chapter{Pendahuluan}

\section{Latar Belakang}
Meningkatnya populasi di indonesia mengakibatkan banyak pelanggan ya-ng mengantri untuk mendapatkan layanan di bank, restoran, rumah sakit, dan tempat penyedia jasa lainnya. Mengantri merupakan kegiatan yang membosankan dan menguras waktu. Panjangnya antrian juga mampu berdampak pada mutu pelayanan di suatu tempat. Pelanggan yang harus menunggu lama berpotensi beralih ke pesaing, atau jika ada urusan lain yang lebih penting, maka pelanggan akan keluar dari tempat antrian, meninggalkan antriannya \cite{khong2017queue}\cite{Ghazal2016}\cite{Uddin2016}. Solusi yang ada pada bank, kantor pos, dan rumah sakit saat ini menggunakan \textit{ticketting} nomor antrian secara manual, dimana antrian yang sedang dilayani ditampilkan di layar pada ruang tunggu. Hal ini kurang efektif karena pelanggan harus berada di ruang tunggu\cite{Ghazal2016}.

Perkembangan teknologi yang cepat mengakibatkan penggunaan perangkat pintar atau \textit{smartphone} merupakan hal lumrah, banyak bermunculan aplikasi antrian virtual seperti Antrique, Qiwee, ExaQue dimana pengguna dapat mengantri dari jarak jauh melalui aplikasi maupun \textit{website}. Para pengguna aplikasi tersebut dapat melakukan hal lain saat mengantri sebelum gilirannya. Namun, aplikasi-aplikasi tersebut memiliki banyak kelemahan seperti \textit{UI/UX} yang tidak bagus, sering \textit{crash} dan \textit{freeze}, tidak ada estimasi waktu antrian, dan masih belum ada yang berfokus ke sektor \textit{food and beverage}.

Oleh karena itu, perlunya dikembangkan sebuah aplikasi yang memiliki fitur yang sama atau lebih dengan menutup kekurangan pada aplikasi tersebut. Pengembangan aplikasi yang direncanakan menggunakan arsitektur monolitik karena mudahnya untuk dibuat dan di-\textit{deploy} secara cepat untuk di iterasikan. Namun, arsitektur monolitik memiliki kelemahan seperti sulitnya untuk di-\textit{maintenance}, \textit{scale}, dan reliabilitas nya. Oleh karena itu, perlu diperhatikan bagaimana cakupan aplikasi kedepannya dan perlunya migrasi ke arsitektur \textit{microservice} \cite{gos2020comparison} \cite{jatkiewicz2023differences}.

Dalam pengembangan aplikasi web, pemilihan bahasa pemrograman untuk digunakan di \textit{backend} sangatlah penting karena dapat mempengaruhi performa aplikasi yang dibangun. Dalam pemilihan bahasa pemrograman \textit{backend}, banyak pilihan yang tersedia seperti PHP, Python, Ruby, PERL, dan banyak lagi. NodeJs merupakan \textit{tools} yang memungkinkan bahasa JavaScript dapat dijalankan pada sisi \textit{backend}. Dalam sisi performa, NodeJs lebih unggul dibanding PHP dan Python dalam sisi kecepatan melayani \textit{request} dari client \cite{William2020} \cite{Odeniran2023}.

NestJs merupakan \textit{framework backend} dari Nodejs yang menggunakan bahasa typescript, dan bisa digunakan untuk pengembangan arsitektur berbasis \textit{microservice} dan monolitik, jadi jika aplikasi dikembangkan pada arsitektur monolitik dapat dengan mudah dimigrasikan ke \textit{microservice}. NestJs juga bisa digunakan bersamaan dengan \textit{framework} PrismaJs untuk mengelola database \cite{NestJS}. PrismaJs merupakan \textit{framework} Object Relational Mapping (ORM) \cite{Prisma}, digunakan untuk mempercepat, dan mempermudah pengembangan aplikasi yang database-nya memiliki relasi yang kompleks dan sulit di-\textit{maintenance} jika menggunakan Structured Query Language (SQL) \cite{Zmaranda2020}.

Implementasi Application Programming Interface (API) yang digunakan adalah Representational State Transfer (RESTful) API, RESTful API adalah arsitektur untuk mempermudah komunikasi client-server agar efektif untuk transaksi data. Namun, pada implementasi RESTful API, ada beberapa hal yang perlu diperhatikan seperti keamanan saat transaksi  atau komunikasi \cite{Beer2018}. Keamanan yang lemah dapat mengakibatkan hacker dapat dengan mudah melakukan \textit{request tampering}, mengambil data pengguna, dan dapat membocorkan data keuangan mitra. \textit{Design pattern} juga perlu diperhatikan dalam penggunaan bahasa untuk API \textit{endpoint} nya agar tidak terjadi \textit{anti pattern}. \textit{Anti pattern} terjadi saat penamaan API tidak sesuai dengan fungsi, atau ada fungsi sejenis tapi penamaannya berbeda jauh. Dengan menghindari \textit{anti pattern}, dapat berakibat ke aplikasi yang lebih mudah di-\textit{sustain} dan di-\textit{maintain} \cite{Aghajani2018} \cite{Alshraiedeh2021}.

Berdasarkan uraian di atas, penelitian ini akan membuat sebuah \textit{backend} aplikasi antrian dengan menggunakan arsitektur monolitik dengan \textit{framework} NestJs dan PrismaJs sebagai \textit{framework} nya. Setelah fitur-fitur aplikasi dibuat, perlu dilakukan unit testing untuk memvalidasi kodingan yang telah ditulis. Hal ini bertujuan untuk meminimalisir bug dan mencegah terjadinya regresi saat fitur baru ditambahkan \cite{runeson2006survey}.

\section{Perumusan Masalah}
Aplikasi antria memerlukan backend developer untuk mengimplementasi-kan fungsi fungsi API dan manajemen database nya. Maka dapat dirumuskan permasalahan sebagai berikut:
\begin{enumerate}
  \item Bagaimana meningkatkan \textit{sustainability} dan \textit{maintainability} pada penggunaan manajemen database.
  \item Bagaimana merancang API yang bebas dari \textit{anti pattern}.
  \item Bagaimana merancang sistem keamanan pada API untuk melayani \textit{request}.
\end{enumerate}


\section{Tujuan}
Tujuan dari pengerjaan Tugas Akhir ini yaitu:
\begin{enumerate}
  \item Mengimplementasikan Prisma ORM untuk meningkatkan \textit{sustainability} dan \textit{maintainability} pada manajemen database.
  \item Membuat dokumentasi API yang dapat dengan mudah di \textit{sustain} dan di \textit{maintain}.
  \item Mengamankan data pengguna dengan menambahkan \textit{anti request tampering} pada setiap header request.
  
\end{enumerate}

\section{Batasan Masalah}
\begin{enumerate}
  \item Hanya berfokus kepada implementasi database menggunakan Prisma ORM.
  \item Berfokus ke bagaimana membuat \textit{endpoint} API yang tidak menimbulkan \textit{anti pattern}.
  \item Implementasi keamanan pada saat penanganan \textit{request} menggunakan JSON Web Token (JWT).
\end{enumerate}

\section{Rencana Kegiatan}
Rencana kegiatan yang akan dilakukan adalah sebagai berikut : Studi Literatur, Pengumpulan data, Perancangan database dengan Prisma, Implementasi RESTful API pada NestJS, Analisa hasil unit testing, dan Penulisan lapoaran.
\section{Jadwal Kegiatan}
\begin{table}[h!]
  \centering
    \caption{Jadwal kegiatan proposal tugas akhir}
  \label{Novella}
  \begin{tabular}{|c|m{2.5cm}|m{0.01cm}|m{0.01cm}|m{0.01cm}|m{0.01cm}|m{0.01cm}|m{0.01cm}|m{0.01cm}|m{0.01cm}|m{0.01cm}|m{0.01cm}|m{0.01cm}|m{0.01cm}|m{0.01cm}|m{0.01cm}|m{0.01cm}|m{0.01cm}|m{0.01cm}|m{0.01cm}|m{0.01cm}|m{0.01cm}|m{0.01cm}|m{0.01cm}|m{0.01cm}|m{0.01cm}|}
    \hline
    \multirow{2}{*}{\textbf{No}} & \multirow{2}{*}{\textbf{Kegiatan}} & \multicolumn{24}{|c|}{\textbf{Bulan ke-}} \\
    \hhline{~~------------------------}
    {} & {} & \multicolumn{4}{|c|}{\textbf{1}} & \multicolumn{4}{|c|}{\textbf{2}} & \multicolumn{4}{|c|}{\textbf{3}} & \multicolumn{4}{|c|}{\textbf{4}} & \multicolumn{4}{|c|}{\textbf{5}} & \multicolumn{4}{|c|}{\textbf{6}}\\
    \hline
    1 & Studi Literatur & \cellcolor{blue!25} & \cellcolor{blue!25} & \cellcolor{blue!25} & \cellcolor{blue!25}& \cellcolor{blue!25} & \cellcolor{blue!25} & \cellcolor{blue!25} & \cellcolor{blue!25}& \cellcolor{blue!25} & \cellcolor{blue!25} & \cellcolor{blue!25} & \cellcolor{blue!25}& \cellcolor{blue!25} & \cellcolor{blue!25} & \cellcolor{blue!25} & \cellcolor{blue!25}& \cellcolor{blue!25} & \cellcolor{blue!25} & \cellcolor{blue!25} & \cellcolor{blue!25}& \cellcolor{blue!25} & \cellcolor{blue!25} & \cellcolor{blue!25} & \cellcolor{blue!25}\\
    \hline
    2 & Pengumpulan Data & \cellcolor{blue!25} & \cellcolor{blue!25} & \cellcolor{blue!25} & \cellcolor{blue!25} & {} & {} & {} & {} & {} & {} & {} & {}& {} & {} & {} & {}& {} & {} & {} & {}& {} & {} & {} & {}\\
    \hline
    3 & Perancangan database dengan prisma &  {} & {} & {} & {}  & \cellcolor{blue!25} & \cellcolor{blue!25} & \cellcolor{blue!25} & \cellcolor{blue!25} & \cellcolor{blue!25} & \cellcolor{blue!25} & \cellcolor{blue!25} & \cellcolor{blue!25} & {} & {} & {} & {}& {} & {} & {} & {}& {} & {} & {} & {}\\
    \hline
    4 & Implementasi RESTful API pada NestJS &  {} & {} & {} & {} & {} & {} & {} & {}& \cellcolor{blue!25} & \cellcolor{blue!25} & \cellcolor{blue!25} & \cellcolor{blue!25} & \cellcolor{blue!25} & \cellcolor{blue!25} & \cellcolor{blue!25} & \cellcolor{blue!25} & {} & {} & {} & {}& {} & {} & {} & {}\\
    \hline
    5 & Analisa hasil unit testing &  {} & {} & {} & {} & {} & {} & {} & {}& {} & {} & {} & {} & \cellcolor{blue!25} & \cellcolor{blue!25} & \cellcolor{blue!25} & \cellcolor{blue!25} & \cellcolor{blue!25} & \cellcolor{blue!25} & \cellcolor{blue!25} & \cellcolor{blue!25} & {} & {} & {} & {}\\
    \hline
    6 & Penulisan Laporan & {} & {} & {} & {} & \cellcolor{blue!25} & \cellcolor{blue!25} & \cellcolor{blue!25} & \cellcolor{blue!25}& \cellcolor{blue!25} & \cellcolor{blue!25} & \cellcolor{blue!25} & \cellcolor{blue!25}& \cellcolor{blue!25} & \cellcolor{blue!25} & \cellcolor{blue!25} & \cellcolor{blue!25}& \cellcolor{blue!25} & \cellcolor{blue!25} & \cellcolor{blue!25} & \cellcolor{blue!25}& \cellcolor{blue!25} & \cellcolor{blue!25} & \cellcolor{blue!25} & \cellcolor{blue!25}\\
    \hline
  \end{tabular}

\end{table}
\newpage

