\chapter{Pendahuluan}
Bagian pendahuluan memuat beberapa substansi sebagai berikut:

\section{Latar Belakang}
Dalam meningkatnya populasi di indonesia, semakin banyak orang orang yang menantri untuk mendapatkan layanan di bank, restoran, rumah sakit, dan tempat penyedia jasa lainnya. 

Mengantri merupakan kegiatan yang membosankan dan menguras waktu. Mengantri juga dapat mempengaruhi kualitas layanan dari suatu tempat. Pengguna yang lama mengantri memiliki kemungkinan untuk pindah ke kompetitor atau jika ada urusan lain yang lebih penting maka akan keluar dari tempat antrian, meninggalkan antrian nya. [9]

Antria merupakan aplikasi untuk mengatasi masalah antrian tersebut dengan menyediakan fitur untuk melihat estimasi waktu, dan pengguna dapat melakukan hal lain tanpa mengantri secara langsung ditempat.

TA dilakukan untuk menjawab keingintahuan mahasiswa mengenai suatu gejala/konsep/dugaan. Kemukakan argumentasi pentingnya dilakukan pengerjaan TA yang diusulkan tersebut dengan menyampaikan hasil-hasil penelitian sebelumnya yang ada pada referensi. Secara umum latar belakang berisi:
• Alasan kenapa kasus/masalah/fenomena tersebut diambil sebagai bahan kajian.
• Apakah ada sebuah konsep baru sebagai hasil penelitian?
• Gap antara kondisi saat ini dengan kondisi yang akan datang (diharapkan).
Uraikan proses-proses yang dilakukan dalam mengidentifikasi masalah yang akan dicari solusinya. Latar belakang minimal 1 halaman.

\section{Perumusan Masalah}
Uraikan permasalahan yang akan dibahas dalam TA dengan mengacu pada latar belakang yang telah disampaikan dan hasil penelitian terdahulu (bila ada). Dalam perumusan masalah dapat dijelaskan definisi, asumsi, dan lingkup yang menjadi batasan TA. Uraian perumusan masalah tidak harus dalam bentuk pertanyaan

\section{Tujuan}
Berikan pernyataan singkat mengenai tujuan TA. Tujuan dapat berupa menguraikan, menerangkan, membuktikan atau menerapkan suatu gejala / konsep / dugaan, atau membuat suatu model. Rumuskan tujuan yang akan dicapai secara spesifik yang merupakan kondisi baru yang diharapkan terwujud setelah TA selesai. Tujuan harus jelas dan dapat diukur, serta berhubungan dengan perumusan masalah yang telah dituliskan. Batasan masalah pun dapat ditambahkan pada bagian ini.

\section{Rencana Kegiatan}
Rencana kegiatan adalah penjelasan mengenai rencana langkah-langkah yang akan dilakukan dalam pengerjaan Tugas Akhir yang memuat: kajian pustaka, cara pengumpulan data (kualitatif, kuantitatif), rancangan penelitian (mencakup prosedur penelitian dan perancangan sistem), cara menguji hasil penelitian (cara penafsiran dan penyimpulan hasil penelitian).


\section{Jadwal Kegiatan}

Jadwal pelaksanaan dibuat berdasarkan metodologi penyelesaian masalah yang digunakan. Bar-chart bisa dibuat per bulan atau per minggu. Dibawah ini adalah merupakan contoh bar-chart:

\begin{table}[h!]
  \centering
    \caption{Jadwal kegiatan proposal tugas akhir}
  \label{Novella}
  \begin{tabular}{|c|m{2.5cm}|m{0.01cm}|m{0.01cm}|m{0.01cm}|m{0.01cm}|m{0.01cm}|m{0.01cm}|m{0.01cm}|m{0.01cm}|m{0.01cm}|m{0.01cm}|m{0.01cm}|m{0.01cm}|m{0.01cm}|m{0.01cm}|m{0.01cm}|m{0.01cm}|m{0.01cm}|m{0.01cm}|m{0.01cm}|m{0.01cm}|m{0.01cm}|m{0.01cm}|m{0.01cm}|m{0.01cm}|}
    \hline
    \multirow{2}{*}{\textbf{No}} & \multirow{2}{*}{\textbf{Kegiatan}} & \multicolumn{24}{|c|}{\textbf{Bulan ke-}} \\
    \hhline{~~------------------------}
    {} & {} & \multicolumn{4}{|c|}{\textbf{1}} & \multicolumn{4}{|c|}{\textbf{2}} & \multicolumn{4}{|c|}{\textbf{3}} & \multicolumn{4}{|c|}{\textbf{4}} & \multicolumn{4}{|c|}{\textbf{5}} & \multicolumn{4}{|c|}{\textbf{6}}\\
    \hline
    1 & Studi Literatur & \cellcolor{blue!25} & \cellcolor{blue!25} & \cellcolor{blue!25} & \cellcolor{blue!25}& \cellcolor{blue!25} & \cellcolor{blue!25} & \cellcolor{blue!25} & \cellcolor{blue!25}& \cellcolor{blue!25} & \cellcolor{blue!25} & \cellcolor{blue!25} & \cellcolor{blue!25}& \cellcolor{blue!25} & \cellcolor{blue!25} & \cellcolor{blue!25} & \cellcolor{blue!25}& \cellcolor{blue!25} & \cellcolor{blue!25} & \cellcolor{blue!25} & \cellcolor{blue!25}& \cellcolor{blue!25} & \cellcolor{blue!25} & \cellcolor{blue!25} & \cellcolor{blue!25}\\
    \hline
    2 & Pengumpulan Data & \cellcolor{blue!25} & \cellcolor{blue!25} & \cellcolor{blue!25} & \cellcolor{blue!25} & {} & {} & {} & {} & {} & {} & {} & {}& {} & {} & {} & {}& {} & {} & {} & {}& {} & {} & {} & {}\\
    \hline
    3 & Analisis dan Perancangan Sistem &  {} & {} & {} & {}  & \cellcolor{blue!25} & \cellcolor{blue!25} & \cellcolor{blue!25} & \cellcolor{blue!25} & \cellcolor{blue!25} & \cellcolor{blue!25} & \cellcolor{blue!25} & \cellcolor{blue!25} & {} & {} & {} & {}& {} & {} & {} & {}& {} & {} & {} & {}\\
    \hline
    4 & Implementasi Sistem &  {} & {} & {} & {} & {} & {} & {} & {}& \cellcolor{blue!25} & \cellcolor{blue!25} & \cellcolor{blue!25} & \cellcolor{blue!25} & \cellcolor{blue!25} & \cellcolor{blue!25} & \cellcolor{blue!25} & \cellcolor{blue!25} & {} & {} & {} & {}& {} & {} & {} & {}\\
    \hline
    5 & Analisa Hasil Implementasi &  {} & {} & {} & {} & {} & {} & {} & {}& {} & {} & {} & {} & \cellcolor{blue!25} & \cellcolor{blue!25} & \cellcolor{blue!25} & \cellcolor{blue!25} & \cellcolor{blue!25} & \cellcolor{blue!25} & \cellcolor{blue!25} & \cellcolor{blue!25} & {} & {} & {} & {}\\
    \hline
    6 & Penulisan Laporan & {} & {} & {} & {} & \cellcolor{blue!25} & \cellcolor{blue!25} & \cellcolor{blue!25} & \cellcolor{blue!25}& \cellcolor{blue!25} & \cellcolor{blue!25} & \cellcolor{blue!25} & \cellcolor{blue!25}& \cellcolor{blue!25} & \cellcolor{blue!25} & \cellcolor{blue!25} & \cellcolor{blue!25}& \cellcolor{blue!25} & \cellcolor{blue!25} & \cellcolor{blue!25} & \cellcolor{blue!25}& \cellcolor{blue!25} & \cellcolor{blue!25} & \cellcolor{blue!25} & \cellcolor{blue!25}\\
    \hline
  \end{tabular}

\end{table}
\newpage

