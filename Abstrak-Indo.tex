\chapter*{Abstrak}

Abstrak berisikan resume yang menggambarkan keseluruhan rencana TA yang akan dikerjakan yang meliputi permasalahan, metodologi dan hipotesis awal. Secara umum poin penting yang harus ada didalam abstrak sebuah proposel TA adalah sebagai berikut:
\begin{enumerate}
    \item Deskripsi singkat permasalahan (1-2 kalimat)
    \item Tujuan utama
    \item Metode yang digunakan atau solusi yang ditawarkan (1-3 kalimat)
    \item Rencana sumber dan/atau jenis data dan/atau studi kasus yang digunakan
    \item Hipotesis awal
\end{enumerate}

Selain itu, pada abstrak harus dituliskan kata kunci atau keyword, yang berisikan kata-kata yang medeskripsikan isi tulisan dan ditulis dengan huruf non kapital. Kata kunci maksimum sebanyak 6 kata.
  
\vspace{0.5 cm}
\begin{flushleft}
{\textbf{Kata Kunci:} keyword-1, keyword-2, keyword-3.}
\end{flushleft}