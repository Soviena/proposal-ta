\chapter*{Abstrak}
Populasi tinggi di indonesia mengakibatkan banyak penyedia layanan jasa seperti bank, dan restoran memiliki antrian yang panjang. Solusi yang sudah ada dengan cara menampilkan urutan antrian  pada layar di ruang tunggu. Dengan membuat aplikasi antrian virtual, diharapkan dapat memperpendek antrian secara fisik, dengan cara antri secara virtual, melihat antrian yang sedang berjalan, dan booking tempat. Pengembangan aplikasi menggunakan framework NestJS, dan PrismaJS dengan menerapkan RESTful API, Object Relational Mapping, dan menghindari Anti-Pattern. Framework NestJS mendukung pembuatan aplikasi ber-arsitektur monolitik dan microservice. Setelah aplikasi dibangun di arsitektur monolitik, aplikasi dapat dengan mudah di migrasikan ke microservice saat penggunaan aplikasi sudah hampir mendekati batas muat pengguna.

\vspace{0.5 cm}
\begin{flushleft}
{\textbf{Kata Kunci:} NestJS, PrismaJS, Anti-Pattern, Antrian, Backend, REST}
\end{flushleft}