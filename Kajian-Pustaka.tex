\chapter{Kajian Pustaka}
 
\section{NodeJs}
Margin yang digunakan adalah batas kiri 1,58 inch (4 cm), sedangkan batas atas, kiri dan bawah adalah 1,18 inch (3 cm). Jenis font yang digunakan adalah Times New Roman. Ukuran font untuk Judul Bab adalah 16 pt, untuk judul sub bab adalah 14 pt, untuk subsub bab, subsubsub bab, dan seterusnya adalah 12 pt. Semua bagian isi proposal TA menggunakan ukuran 12 pt dengan 1 spasi. 
Proposal TA dibuat dengan bantuan komputer menggunakan pencetak (printer) dengan tinta berwarna hitam. Untuk gambar-gambar berwarna proses pencetakan disesuaikan dengan kebutuhan tingkat kepentingan Tema yang akan dikerjakan. 

\subsection{NestJs}

Penulisan proposal TA harus mengikuti kaidah penulisan yang layak seperti 
\begin{enumerate}
    \item penggunaan bahasa dan istilah yang baku dengan singkat dan jelas,
    \item mengikuti kelaziman penulisan pada disiplin keilmuan yang diikuti.
\end{enumerate}
Bahasa Indonesia yang digunakan dalam naskah proposal TA harus bahasa Indonesia dengan tingkat keresmian yang tinggi dengan menaati kaidah tata bahasa resmi. Kalimat harus utuh dan lengkap. Pergunakanlah tanda-baca seperlunya dan secukupnya agar dapat dibedakan anak kalimat dari kalimat induknya, kalimat keterangan dari kalimat yang diterangkan, dan sebagainya. 
Kata ganti orang, terutama kata ganti orang pertama (saya dan kami), tidak digunakan, kecuali dalam kalimat kutipan. Susunlah kalimat sedemikian rupa sehingga kalimat tersebut tidak perlu memakai kata ganti orang. Suatu kata dapat dipisahkan menurut ketentuan tata bahasa. Kata terakhir pada dasar halaman tidak boleh dipotong. Pemisahan kata asing harus mengikuti cara yang ditunjukkan dalam kamus bahasa asing tersebut. Gunakanlah buku Pedoman Umum Ejaan Bahasa Indonesia Yang Disempurnakan, Pedoman Umum Pembentukan Istilah, Kamus Besar Bahasa Indonesia, 


\subsection{PrismaJs}
Menurut paper Kentang \cite{Kentang}, prsamaan SWE adalah
Berikut diberikan persamaan pengatur dari persamaan gelombang pada gitar

\begin{equation}\label{Pers1}
    a=b+U^{n+1}_{i+1}
\end{equation}
Persamaan (\ref{Pers1}) jadsbahdhavhdvah ajdbajdb
\begin{equation}\label{nama-rumus}
    \int_0^1 \frac{f(x)}{g(x)}\ {\rm dx}=\sin x
\end{equation}

\begin{equation}\label{nama-rumus1}
   \alpha \times \beta =\gamma^{3\alpha}
\end{equation}

\begin{figure}[h!]
    \centering
    \includegraphics[scale=0.1]{Tel-U-Logo.png}
    \caption{Caption}
    \label{fig:my_label1}
\end{figure} 

Rumus (\ref{nama-rumus}) merupakan \textit{contoh} persamaan matematika. persamaan matematika diatas diberi nama \textbackslash label\{nama-rumus\}. dengan $\alpha=\gamma \times 100$

\begin{figure}[h!]
    \centering
    \includegraphics[scale=0.3]{Tel-U-Logo.png}
    \caption{Caption}
    \label{fig:my_label}
\end{figure}

Lihat \textit{pada} Gambar \ref{fig:my_label}

\subsection{Object Relational Mapping}
Contoh pustaka prosiding \cite{doyen2014explicit}, jurnal \cite{gunawan2015hydrostatic} dan buku \cite{toro2013riemann}. Atau dapat juga mengguanakan dua pustaka atau lebih dalam \cite{gunawan2015hydrostatic,toro2013riemann}.

Daftar pustaka berisikan daftar referensi yang digunakan dalam pembuatan buku TA ini, dimana minimal terdapat 10 referensi yang digunakan dan seluruh referensi yang ada tercatat diacu dalam buku TA. Sedikitnya 3 referensi yang dijadikan sebagai basis mendapatkan gap/peluang penelitian berasal dari publikasi dalam kurun waktu 5 tahun terakhir, dan termasuk dalam jurnal terindeks Scopus/WoS dan/atau SINTA 1 atau 2.

\subsection{Arsitektur Monolitik}

\subsection{JSON Web Token}

\subsection{Anti Pattern}

\subsection{RESTful API}

% Anti Pattern Website
% https://marcelocure.medium.com/rest-anti-patterns-b128597f5430
% https://punchpicks.medium.com/rest-api-design-antipatterns-76a52e1376c2